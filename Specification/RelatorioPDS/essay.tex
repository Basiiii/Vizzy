%%%%%%%%%%%%%%%%%%%%%%%%%%%%%%%%%%%%%%%%%
% Thin Sectioned Essay
% LaTeX Template
% Version 1.0 (3/8/13)
%
% This template has been downloaded from:
% http://www.LaTeXTemplates.com
%
% Original Author:
% Nicolas Diaz (nsdiaz@uc.cl) with extensive modifications by:
% Vel (vel@latextemplates.com)
%
% License:
% CC BY-NC-SA 3.0 (http://creativecommons.org/licenses/by-nc-sa/3.0/)
%
%%%%%%%%%%%%%%%%%%%%%%%%%%%%%%%%%%%%%%%%%

%----------------------------------------------------------------------------------------
%	PACKAGES AND OTHER DOCUMENT CONFIGURATIONS
%----------------------------------------------------------------------------------------

\documentclass[a4paper, 12pt]{article} % Font size (can be 10pt, 11pt or 12pt) and paper size (remove a4paper for US letter paper)
\usepackage[portuguese]{babel}

\usepackage[protrusion=true,expansion=true]{microtype} % Better typography
\usepackage{graphicx} % Required for including pictures
\usepackage{wrapfig} % Allows in-line images

\usepackage{mathpazo} % Use the Palatino font
\usepackage[T1]{fontenc} % Required for accented characters
\linespread{1.05} % Change line spacing here, Palatino benefits from a slight increase by default

\makeatletter
\renewcommand\@biblabel[1]{\textbf{#1.}} % Change the square brackets for each bibliography item from '[1]' to '1.'
\renewcommand{\@listI}{\itemsep=0pt} % Reduce the space between items in the itemize and enumerate environments and the bibliography

\renewcommand{\maketitle}{
\begin{titlepage}
    \begin{center}
        \vspace*{1cm}
        \includegraphics[width=0.35\textwidth]{logo_no_bg.png}\\[1cm] % Logo
        {\Huge\textbf{Vizzy - Plataforma Comunitária}}\\[0.5cm] % Main Title
        {\Large Projeto de Desenvolvimento de Software}\\[2cm] % Subtitle
        {\large \textsc{
        Enrique Rodrigues Nº28602 \\
        José Alves Nº27967 \\
        Diogo Machado Nº26042 \\
        Diogo Abreu Nº27975 \\
        André Silva Nº27965}}\\[0.5cm] % Authors
        {\textit{Instituto Politécnico do Cávado e do Ave}}\\[1.5cm] % Institution
        {\large \today} % Date
        \vfill
        % \textbf{Keywords:} lorem, ipsum, dolor, sit amet, lectus % Keywords
    \end{center}
\end{titlepage}
}
\makeatother

%----------------------------------------------------------------------------------------
%	TITLE
%----------------------------------------------------------------------------------------

\title{\textbf{Vizzy - Plataforma Comunitária}\\ % Title
	Projeto de Desenvolvimento de Software} % Subtitle

\author{
	\textsc{Enrique Rodrigues Nº28602}, \\
	\textsc{José Alves}, \\
	\textsc{Diogo Machado}, \\
	\textsc{Diogo Abreu Nº27975}, \\ 
	\textsc{André Silva} \\
	\textit{Instituto Politécnico do Cávado e do Ave}
}

\date{\today} % Date

%----------------------------------------------------------------------------------------

\begin{document}

\maketitle % Print the title section

%----------------------------------------------------------------------------------------
%	ABSTRACT AND KEYWORDS
%----------------------------------------------------------------------------------------

%\renewcommand{\abstractname}{Summary} % Uncomment to change the name of the abstract to something else

% \begin{abstract}
% Morbi tempor congue porta. Proin semper, leo vitae faucibus dictum, metus mauris lacinia lorem, ac congue leo felis eu turpis. Sed nec nunc pellentesque, gravida eros at, porttitor ipsum. Praesent consequat urna a lacus lobortis ultrices eget ac metus. In tempus hendrerit rhoncus. Mauris dignissim turpis id sollicitudin lacinia. Praesent libero tellus, fringilla nec ullamcorper at, ultrices id nulla. Phasellus placerat a tellus a malesuada.
% \end{abstract}

% \hspace*{3,6mm}\textit{Keywords:} lorem , ipsum , dolor , sit amet , lectus % Keywords

% \vspace{30pt} % Some vertical space between the abstract and first section

%----------------------------------------------------------------------------------------
%	DOCUMENT BODY
%----------------------------------------------------------------------------------------

\newpage
\section*{Introdução}

No mundo atual, a economia colaborativa tem ganhado cada vez mais destaque, incentivando práticas como a troca, venda e empréstimo de produtos entre indivíduos de uma mesma comunidade. Com base nessa tendência, este projeto propõe o desenvolvimento de uma plataforma que facilite essas transações de forma organizada e segura. 
Através da marcação de encontros dentro da comunidade, os utilizadores poderão partilhar recursos, reduzindo desperdícios e promovendo um consumo mais sustentável.
Este relatório apresenta os principais elementos do projeto, incluindo diagramas UML e mockups, que ilustram a estrutura e funcionalidades da plataforma.

%------------------------------------------------

\section*{Contexto e Introdução do Problema}

Em muitas comunidades, o consumo desenfreado e a dificuldade de acesso a determinados produtos criam desafios tanto econômicos quanto ambientais. 

Muitas vezes, itens de valor são descartados ou ficam inutilizados por falta de uso, enquanto outras pessoas poderiam se beneficiar deles. Paralelamente, a aquisição de novos produtos pode ser um obstáculo devido a limitações financeiras ou dificuldade de disponibilidade.

Nesse contexto, um sistema de troca, venda e empréstimo de produtos dentro de uma comunidade pode se tornar uma solução eficiente para otimizar o uso dos recursos, reduzir desperdícios e fortalecer os laços entre os membros da comunidade. 

Esse modelo permite que os indivíduos tenham acesso a itens de que necessitam sem necessariamente precisar comprá-los novos, ao mesmo tempo em que promovem a circularidade dos bens.

\section*{Descrição do Problema e Motivação para a Solução}

O problema central reside na falta de um sistema organizado e acessível para facilitar a troca, venda e empréstimo de produtos dentro de uma comunidade. 

Atualmente, muitas pessoas não sabem onde ou como oferecer seus itens para troca ou venda, resultando em desperdício e em um ciclo de consumo ineficiente. 

Além disso, a aquisição de produtos novos pode ser um desafio financeiro para algumas famílias, tornando a economia colaborativa uma alternativa atraente.

A solução proposta é a criação de um sistema onde os membros de uma comunidade possam marcar encontros para trocar, vender ou emprestar produtos de forma organizada e segura. 

Além disso, o sistema permitiria que um membro da comunidade fizesse uma marcação antecipada de quando necessitará de determinado produto, garantindo que itens parados sejam utilizados mais vezes e de forma mais eficiente. 

Esse sistema permitiria uma gestão eficiente dos recursos, promovendo a reutilização de itens e reduzindo a necessidade de compras desnecessárias. Além do impacto econômico positivo, a iniciativa também traria benefícios ambientais ao reduzir o descarte de produtos ainda em boas condições.

Dessa forma, ao estabelecer uma plataforma de fácil acesso para troca, venda e empréstimo, seria possível fomentar a cooperação entre os membros da comunidade, fortalecer o senso de pertencimento e promover um consumo mais consciente e sustentável.

\section*{Section Name}

Cras gravida, est vel interdum euismod, tortor mi lobortis mi, quis adipiscing elit lacus ut orci. Phasellus nec fringilla nisi, ut vestibulum neque. Aenean non risus eu nunc accumsan condimentum at sed ipsum.
\begin{wrapfigure}{l}{0.4\textwidth} % Inline image example
\begin{center}
\includegraphics[width=0.38\textwidth]{fish.png}
\end{center}
\caption{Fish}
\end{wrapfigure}
Aliquam fringilla non diam sed varius. Suspendisse tellus felis, hendrerit non bibendum ut, adipiscing vitae diam. Lorem ipsum dolor sit amet, consectetur adipiscing elit. Nulla lobortis purus eget nisl scelerisque, commodo rhoncus lacus porta. Vestibulum vitae turpis tincidunt, varius dolor in, dictum lectus. Aenean ac ornare augue, ac facilisis purus. Sed leo lorem, molestie sit amet fermentum id, suscipit ut sem. Vestibulum orci arcu, vehicula sed tortor id, ornare dapibus lorem. Praesent aliquet iaculis lacus nec fermentum. Morbi eleifend blandit dolor, pharetra hendrerit neque ornare vel. Nulla ornare, nisl eget imperdiet ornare, libero enim interdum mi, ut lobortis quam velit bibendum nibh.

Morbi tempor congue porta. Proin semper, leo vitae faucibus dictum, metus mauris lacinia lorem, ac congue leo felis eu turpis. Sed nec nunc pellentesque, gravida eros at, porttitor ipsum. Praesent consequat urna a lacus lobortis ultrices eget ac metus. In tempus hendrerit rhoncus. Mauris dignissim turpis id sollicitudin lacinia. Praesent libero tellus, fringilla nec ullamcorper at, ultrices id nulla. Phasellus placerat a tellus a malesuada.

%------------------------------------------------

\section*{Conclusion}

A plataforma proposta tem como objetivo criar um ambiente seguro e eficiente para a troca, venda e empréstimo de produtos dentro de uma comunidade. Ao permitir a marcação de encontros para a realização das transações, o projeto fomenta a interação entre os membros e incentiva práticas de consumo consciente. 
Com a definição clara dos requisitos e o suporte de diagramas UML e mockups, este relatório serve como um guia para o desenvolvimento da solução, garantindo que todos os aspectos essenciais sejam considerados.


%----------------------------------------------------------------------------------------
%	BIBLIOGRAPHY
%----------------------------------------------------------------------------------------

\bibliographystyle{unsrt}

\bibliography{sample}

%----------------------------------------------------------------------------------------

\end{document}